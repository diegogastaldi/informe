\lhead{\emph{Limitaciones}}

\chapter{Limitaciones}

La herramienta que se detalla en este informe presenta diversas limitaciones, que pueden clasificarse en dos grupos: aquellas provenientes de las aplicaciones que usa, en especial {\tt FlowDroid} y {\tt Epicc}; y aquellas limitaciones propias de la herramienta.
En el primer grupo, una limitaci�n importante es la falta de solidez del an�lisis producida por las llamadas recursivas y el c�digo nativo, ya que estas no son tratadas de manera eficaz por {\tt Epicc} ni por {\tt FlowDroid}. {\tt FlowDroid} analiza las llamadas que invocan a c�digo nativo de la siguiente manera: si los datos de entrada estaban {\it contaminados} (forma por la cual {\tt Flowdroid} determina el flujo de la informaci�n) antes de la llamada, entonces se determina que todos los argumentos de llamada y cualquier valor de retorno est�n contaminados. Este manejo no es s�lido; no analiza los flujos reales en el c�digo nativo invocado por lo cual la informaci�n obtenida puede generar muchos {\it falsos positivos}. Tampoco son consideradas las fugas de informaci�n que involucre {\it multithread} (hilos de ejecuci�n).\par

{\tt Didfail} solo sigue el flujo de la informaci�n si la informaci�n es transmitida a trav�s de {\it Intents}. Por ejemplo, si dos aplicaciones se env�an  informaci�n utilizando {\it campos est�ticos} compartidos como medio de comunicaci�n {\tt Didfail} no detectar� el flujo de informaci�n. Adem�s, {\tt Didfail} tambi�n detecta posibles flujos de informaci�n que son imposibles que sucedan en realidad.\par

En el segundo grupo de limitaciones se debe mencionar que el usuario debe dar la definici�n de los niveles de seguridad  de manera correcta. Para esto, el usuario debe tener conocimiento sobre relaciones de orden y configuraci�n de entornos mediante el uso de archivos de configuraci�n. Adem�s, el usuario debe conocer las variables asociadas a cada uno de los recursos del dispositivo {\tt Android} para poder asociarles los niveles de seguridad deseados. Es necesario resaltar que la cantidad de estas variables es considerable.   
Tanto la definici�n de los niveles de seguridad, como la asignaci�n de niveles a cada uno de los recursos del dispositivo m�vil, son pasos necesarios para poder utilizar de manera correcta la herramienta. \par 

Cabe resaltar que las limitaciones mencionadas en el segundo grupo se minimizan para el usuario si  utiliza la interfaz que provee la herramienta.\par

Las pruebas de la herramienta fueron realizadas utilizando aplicaciones que podemos denominar de {\it juguete}, es decir, aplicaciones desarrolladas exclusivamente para probar esta herramienta. Es necesario realizar pruebas a mayor escala, utilizando aplicaciones de uso cotidiano, que pertenezcan al mundo real, para poder comprobar de manera mas certera el rendimiento y la exactitud de los resultados.\par