\lhead{\emph{Limitaciones}}

\chapter{Limitaciones}

La herramienta que se detalla en este informe presenta diversas limitaciones, que pueden clasificarse en dos grupos: aquellas provenientes de las aplicaciones que usa (FlowDroid, Epicc, entre otras); y aquellas limitaciones propias de la herramienta.
En el primer grupo, una limitaci�n importante es la falta de solidez del an�lisis producida por las llamadas recursivas y el c�digo nativo, ya que estos no son tratadas de manera eficaz por Epicc ni FlowDroid. FlowDroid analiza las llamadas que invocan a c�digo nativo de la siguiente manera: si los datos de entrada estaban ``contaminado" (forma por la cual Flowdroid determina el flujo de la informaci�n) antes de la llamada, entonces se determina que todos los argumentos de llamada y cualquier valor de retorno est�n contaminados. Este manejo no es s�lido; no analiza el c�digo nativo en el destinatario de la llamada. Tampoco son consideradas las fugas de informaci�n que involucre multithread (hilos de ejecuci�n).\par
Didfail solo sigue el flujo de la informaci�n si �sta es transmitida a trav�s de Intents, por lo que dos aplicaciones pueden enviarse informaci�n de manera tal que el an�lisis no lo detecte utilizando, por ejemplo, campos est�ticos compartidos como medio de comunicaci�n. Adem�s, Didfail tambi�n detecta posibles flujos de informaci�n, que son imposibles que sucedan en realidad.\par
En el segundo grupo de limitaciones, la definici�n de los niveles de seguridad es necesaria para poder utilizar de manera correcta la herramienta, y para lograr esto, el usuario debe tener conocimiento sobre relaciones de orden, configuraci�n de entornos mediante el uso de archivos de configuraci�n. Adem�s, deber� conocer las variables asociadas a los recursos para poder asociarles los niveles de seguridad, esto �ltimo solo en los casos de que no se utilice la interfaz que provee la herramienta.\par
Las pruebas fueron realizadas utilizando aplicaciones que podemos denominar de "juguete", es decir, aplicaciones desarrolladas exclusivamente para probar esta herramienta. Es necesario realizar pruebas a mayor escala, utilizando aplicaciones de uso cotidiano, que pertenezcan al mundo real, para poder comprobar de manera mas certera el rendimiento y la exactitud de los resultados.\par