\lhead{\emph{Conclusi�n y futuros trabajos}}

\chapter{Conclusi�n y futuros trabajos}

El an�lisis est�tico inter e intraprocedural que lleva a cabo el prototipo descripto en este informe permite, no solo, determinar el flujo de la informaci�n, sino tambi�n,  garantizar la confidencialidad e integridad de la informaci�n manipulada por programas {\tt Android} siguiendo la configuraci�n (pol�tica de seguridad) que establezcan los usuarios de acuerdo a sus necesidades. Este prototipo tambi�n permite detectar {\it sources} y/o {\it sinks} potencialmente peligrosos; es decir, la informaci�n generada por la herramienta brinda a los usuarios informaci�n de gran relevancia para determinar niveles de seguridad de {\it sources} y {\it sinks} que no fueron considerados en primer instancia por los usuarios (o no fueron considerados sensibles desde el punto de vista de la seguridad) al definir su pol�tica de seguridad.\par

La construcci�n de este prototipo clarifica la viabilidad, eficiencia y eficacia del uso de an�lisis est�tico para la verificaci�n de confidencialidad e integridad de la informaci�n. \par

Es viable porque la implementaci�n as� lo demuestra, si bien posee limitaciones y requiere refinamientos tales como la adici�n de formas en la que el flujo de informaci�n puede transmitirse, como por ejemplo los {\it campos est�ticos}, bases de datos {\tt SQLite} y {\it SharedPreferences}.\par

En cuanto a la eficiencia, esto es un punto cr�tico para los dispositivos m�viles, ya que, si bien cada vez son mas potentes y tiene mas recursos, el an�lisis completo requiere de un gran uso de recursos. Realizar el an�lisis completo en un dispositivo m�vil ser�a completamente ineficiente y costoso. Como alternativa a ello se puede dividir el an�lisis en etapas (como las mencionadas en el cap�tulo de dise�o e implementaci�n de la herramienta) donde en el repositorio de aplicaciones se lleve a cabo el an�lisis m�s {\it pesado} (el cual se corresponde con la etapa n�mero uno) y en el dispositivo solo se realice la segunda etapa del an�lisis (el cual requiere que se introduzca la pol�tica de seguridad y de acuerdo a ella se complete el an�lisis). \par

Por otro lado, si bien esta herramienta requiere de refinamiento como se mencion� anteriormente, teniendo en cuenta sus limitaciones y objetivos, realiza un an�lisis eficaz, detectando los flujos para los cuales esta capacitada para hacerlo y omitiendo aquellos en los que no se tuvieron en cuenta, como por ejemplo, los {\it campos est�ticos}.\par