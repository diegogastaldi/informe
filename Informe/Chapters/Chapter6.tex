\lhead{\emph{Conclusi�n y futuros trabajos}}

\chapter{Conclusi�n y futuros trabajos}

El an�lisis est�ticos inter e intraprocedural que lleva a cabo el prototipo descripto en este informe permite determinar el flujo de la informaci�n y esto a su vez puede ser utilizado para garantizar la confidencialidad e integridad de la informaci�n manipulada por programas Android siguiendo la configuraci�n que establezcan los usuarios de acuerdo a sus necesidades. Este prototipo tambi�n permite detectar sources y/o sinks potencialmente peligrosos y ayuda a determinar niveles de seguridad de �stos para aquellos que no tienen nivel definido por el usuario.\par
La construcci�n de este prototipo clarifica la viabilidad, eficiencia y eficacia del uso de an�lisis est�tico para la verificaci�n de confidencialidad e integridad de la informaci�n. \par
Es viable porque la implementaci�n as� lo demuestra, si bien posee muchas limitaciones y requiere refinamientos tales como la adici�n de formas en la que el flujo de informaci�n puede transmitirse, como por ejemplo los campos est�ticos, bases de datos SQLite y SharedPreferences.\par
En cuanto a la eficiencia, esto es un punto critico para los dispositivos m�viles ya que si bien cada vez son mas potentes y tiene mas recursos, el an�lisis completo requiere de un gran uso de estos y realizarlo completo en un dispositivo m�vil ser�a completamente ineficiente. Como alternativa a ello se puede dividir el an�lisis en etapas como las mencionadas en el capitulo de dise�o e implementaci�n de la herramienta, donde el repositorio de aplicaciones lleve a cabo el an�lisis mas "pesado" sobre las aplicaciones, el cual se corresponde con la etapa n�mero uno, y en el dispositivo solo se realiza la segunda etapa del an�lisis, en la cual se requiere que introduzca su configuraci�n preferida y de acuerdo a ella se complete el an�lisis. \par
Por otro lado, si bien esta herramienta requiere de refinamiento como se mencion� antes, teniendo en cuenta sus limitaciones y objetivos, realiza un an�lisis eficaz, detectando los flujos para los cuales esta capacitada para hacerlo y omitiendo aquellos en los que no se tuvieron en cuenta como por ejemplo los campos est�ticos.\par